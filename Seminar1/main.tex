\documentclass[12pt]{article}
\usepackage[utf8]{inputenc}
\usepackage{parskip}
\usepackage{markdown}
\usepackage{hyperref}
\usepackage{listings}
\usepackage{color}
\usepackage[subtle]{savetrees}
\usepackage{verbatim}
\usepackage{blindtext}

\title{\vspace{-2cm} \textbf{Session 1 - Hello World!} \\ UCAS Program 2020}

\author{Chloe Lau}
\date{August 2020}

\begin{document}
\setlength{\parindent}{4ex}
\setlength{\parskip}{1em}

\maketitle

\section{Introduction}
Hello! I am Chloe, a Second Year Computing Student at Imperial College London. I am your UCAS Tutor!

Just a few details:
\begin{itemize}
    \item I am based in London, so my timezone would be British Summer Time (GMT+1).
    \item Yes, I am an alumni of CSFC, I was in your seats 2 years ago.
    \item You can contact me via this \href{mailto:ucastutor19@ccoex.com}{email}.
    \item I am a Committee of Imperial's DoCSoc (Department of Computing Society), throw me questions about it!
    \item My own personal statement can be found  \href{https://drive.google.com/file/d/1TmUbngo7RldUyzRq49FYU50ye2DMcl5a/view?usp=sharing}{here}. Please do not copy it or share it to your friends!
    \item I am obsessed with \LaTeX{}, come ask me about it if you want to learn it or have any issues!
\end{itemize}

In the following 10 days, I will be helping with your personal statements, interview skills, and anything interesting! All of the seminar's notes will be written in \LaTeX{} and distributed mostly directly in the seminars (with exceptions when some resources are needed).

\section{Hello World?}
Tell me more about yourself:)
\begin{itemize}
    \item Who are you?
    \item Where are you from?
    \item What areas are you interested?
    \item A random fact about yourself?
\end{itemize}

\section{Let's Go On!}

So now that you know who I am, this is the age old question(s) that all interviews will inevitably ask:

\subsection{Why Computer Science?}
Have you ever thought of why? Is it a very generic answer? Anything that sparked your interests in Computer Science?

Okay now that you have finished tackling that question, here's another one:

\subsection{Why [inserts University Name]?}
How did you choose your university? Rankings? Modules? Social life? Are you sure this university suits you?

Wow so you finished two hard questions! Let's move on to some more:

\subsection{What are your interests in Computer Science? Any research areas you are particularly interested?}
What is your interested area? Robotics? AIML (yes this is a buzzword, really?)? What inspired you to be interested in it? Is there anything that's not a buzzword but you are really interested? 

Anyways let's continue:

\subsection{Is HTML and CSS a language?}
Hah, no this is not a serious question.

NEXT!

\subsection{Are there any interesting books or articles you have recently read?}
CODE: The Hidden Language of Computer Hardware and Software? Algorithms to Live By (I added this to the CS recommended reading list a few years back)? Computational Fairy Tales? Did you actually read through it?

Good reads are good things to mention in interviews, but hey! Don't forget we are doing questions now!

\subsection{Did you work on anything interesting at all? Projects? Courses?}
Made your own chat bot? Did CS50? Anything special?

Last but not least, here is the last question:

\subsection{So... What did you learn from all your preparation?}
This is the most important thing, you could have done loads of preparation, but how many of them are relevant? How many of them brought you knowledge that you find interesting? 

Universities not only care about the quantity of your supercurriculars and activities, they care about the \textbf{quality}!

Remember how Mr. Kilner used to say how reflection is essential to tackling STEP problems? It is essential for all sorts of problems. Now try and think about problems in a whole new way, with reflections!
\end{document}
